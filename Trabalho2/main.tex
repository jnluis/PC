\documentclass[a4paper, titlepage, oneside, 11pt]{article}
\usepackage[total={6in, 8in}, margin=2.5cm, headsep=2cm, top=3cm ]{geometry}
\usepackage[portuguese]{babel}
\usepackage{graphicx}
\usepackage{parskip}
\usepackage[acronym]{glossaries}
\graphicspath{{images/}}
\usepackage{indentfirst}
\usepackage{setspace}
\usepackage{xcolor}
\definecolor{LightGray}{gray}{0.9}
\usepackage{float}
\usepackage{fontspec}
\usepackage{fancyvrb}
\usepackage{fancyhdr}
\pagestyle{fancy}
\usepackage{fvextra}
\usepackage{hyperref}
\usepackage{minted}
\usepackage{caption}
\hypersetup{colorlinks,linkcolor={black},citecolor={blue},urlcolor={blue}}
\setstretch{1.5}

\fancyheadoffset{0pt}

\fancyhead[L]{Psicologia e Cibersegurança \\[1ex] Trabalho 2\\} 
\fancyhead[R]{\includegraphics[width=0.3\linewidth]{images/deti.png}\\} 
\fancyfoot[C]{\thepage}
\fancyfoot[L]{João Nuno da Silva Luís}
\fancyfoot[R]{16/12/2025}


\usepackage{afterpage}
\usepackage{minted}

\newcommand\blankpage{%
    \null
    \thispagestyle{empty}%
    \addtocounter{page}{-1}%
    \newpage}

\newacronym{uc}{UC}{Unidade Curricular}

\begin{document}

\pagenumbering{arabic}

\section*{\centering{\Large{Opção A - Em que \textit{Phishing} cairia eu?}}}

Neste trabalho pretende-se refletir um pouco sobre como é que alguém da área de TI, mais especificamente de cibersegurança, poderia ser vítima de um ataque de \textit{phishing}.

Apesar de profissionais e estudantes de TI terem, em média, maior consciencialização em cibersegurança, vários estudos demonstram que nenhum grupo está completamente imune a este ataque de engenharia social, sobretudo quando estes são altamente personalizados ou contextualizados.

Um estudo conduzido por Daengsi et al. (2021) \cite{daengsi2021} analisou o impacto da consciencialização em cibersegurança através de simulações de phishing numa organização financeira. Os autores compararam funcionários de departamentos técnicos (como TI) com departamentos de natureza social (como Recursos Humanos e Jurídico). Os resultados mostraram que, embora os colaboradores de TI apresentassem inicialmente maior capacidade de deteção de phishing, ambos os grupos eram vulneráveis. Após ações de formação e simulações, verificou-se uma melhoria significativa nos dois grupos, especialmente nos colaboradores não técnicos. Este estudo evidencia que a formação contínua e a exposição a cenários realistas são fundamentais para reduzir o sucesso de ataques de phishing, independentemente do perfil técnico do utilizador.

De forma complementar, Meyers et al. (2018) \cite{meyers2018} estudaram a formação de estudantes de cibersegurança na criação de ataques de \textit{spear phishing} através de um processo estruturado denominado SiEVE (Social Engineering Vulnerability Evaluation). O estudo demonstrou que alunos que seguiram este processo conseguiram criar emails de \textit{spear phishing} mais convincentes e eficazes. Estes resultados mostram que ataques bem-sucedidos dependem fortemente do contexto e da personalização da mensagem, reforçando a ideia de que utilizadores, mesmo com formação técnica, podem ser enganados quando confrontados com mensagens altamente credíveis.

Em conjunto, estes trabalhos reforçam a importância da consciencialização contínua em cibersegurança e demonstram que o \textit{phishing}, especialmente o \textit{spear phishing}, continua a ser uma ameaça relevante e eficaz.


Antes de elaborar em que tipos de \textit{phishing} que poderia cair, convidava o professor a fazer um pequeno \textit{quiz} para testar as suas capacidades, disponível em \href{https://phishingquiz.withgoogle.com/}{https://phishingquiz.withgoogle.com/}. Eu obtive o seguinte resultado: 

\begin{figure}[H]
	\centering
	\includegraphics[width=1\linewidth]{images/phishing.png}
	\caption{Resultados do questionário de phishing}
	\label{fig:phishing}
\end{figure}

Como é possível ver, não acertei em todas as perguntas devido a um dos aspetos mais importantes no phishing na minha visão: \textbf{o contexto}, já que em ambas as perguntas erradas é importante perceber se de facto existia algum processo de compra/encomenda em curso. 

No entanto, com base neste questionário, é possível identificar algumas técnicas comuns utilizadas por atacantes de \textit{phishing}:

\begin{itemize}
	\item Uso de domínios muito semelhantes aos legítimos, com pequenas alterações (ex:substituição de caracteres);
	\item Criação de um sentido de urgência ou escassez de tempo para incentivar uma ação imediata;
	\item Exploração de temas relevantes para o utilizador, como atualizações de \textit{software} ou problemas de armazenamento.
\end{itemize}

Adicionalmente, verifica-se que o uso de dispositivos móveis aumenta a exposição ao \textit{phishing}, uma vez que o utilizador tende a clicar diretamente nos \textit{links} e não a copiá-los para um separador novo, sem analisar o endereço de destino, ao contrário do que acontece num computador em que basta dar \textit{hover} sobre o mesmo.

Tendo em conta os tipos de \textit{phishing} abordados na Unidade Curricular, o ataque ao qual considero estar mais vulnerável é o \textit{spear phishing}, mais especificamente o \textit{whaling}. Neste cenário, o atacante faria passar-se não pelo Reitor da UA, mas pelo meu orientador de bolsa e tese, o Professor Doutor João Paulo Barraca.

Um exemplo plausível seria um convite falso para uma reunião no Microsoft Teams, que é bastante comum em ambos os trabalhos desenvolvidos. Outra possibilidade seria um email relacionado com a plataforma NextCloud, utilizada para armazenamento e partilha de ficheiros no âmbito da bolsa. Neste segundo caso, o fator de urgência poderia ser explorado através da proximidade de prazos para entrega de indicadores de progresso que ocorrem mensalmente, aumentando a probabilidade de uma ação impulsiva.

De seguida, apresento um mail exemplo para o primeiro caso:

\begin{figure}[H]
	\centering
	\includegraphics[width=1\linewidth]{images/Mail1}
	\caption{Exemplo de mail de phishing 1}
	\label{fig:mail1}
\end{figure}

\vspace{-10pt}

Neste caso, o atacante faria-se passar pelo professor com um \textit{mail} semelhante ao legítimo \\ (jpbarracas@ua.pt), colocando em CC um mail semelhante ao do co-orientador. De resto, a complexidade estaria em replicar visualmente o convite que é enviado pelo Microsoft Teams.

Já para o segundo caso, o \textit{mail} poderia ser da seguinte forma:


\begin{minted}[
	baselinestretch=1.2,
	bgcolor=LightGray,
	fontsize=\footnotesize,
	breaklines,
	escapeinside=||
	]{text}
	De: João Barraca (mail falso semelhante ao do Exemplo1)
	Para info@ccc-centro.pt
	Data: 30-10-2025
	
	Assunto: C-Network \|| Toolkit \|| Validação de Indicadores CCC Centro \|| Outubro 2025
	
	Boa tarde,
	Preciso de enviar os KPIs atualizados até amanhã. Por favor atualizem o |\underline{\textcolor{blue}{Excel}}| do NextCloud.
	Cumprimentos,
\end{minted}

\begin{minipage}[t]{0.5\textwidth}
	\begin{minted}[
		baselinestretch=1.2,
		bgcolor=LightGray,
		fontsize=\footnotesize,
		breaklines,
		escapeinside=||
		]{text}
		
		
		João Paulo Barraca
		Associate Professor
		Cybersecurity Office
		DETI / GCS / UNAVE / IT
		University of Aveiro
		tel. +351 927 036 039
		ext. 90140
		web. www.ua.pt/pt/p/10333322
	\end{minted}
\end{minipage}%
\hfill
\begin{minipage}[t]{0.5\textwidth}
	\vspace{0pt}
	\begin{figure}[H]
		\centering
		\includegraphics[width=0.4\linewidth]{images/Mail2}
		\caption{Assinatura do email}
		\label{fig:assinatura}
	\end{figure}
\end{minipage}

\vspace{5pt}

Neste caso, para o \textit{mail} ter maior probabilidade de ser clicado por mim, deveria haver um discurso informal e bastante direto, característico do professor. Além disso, a forma como os \textit{mails} vêm assinados também é algo que costumo ter em atenção mas é fácil de fazer uma assinatura como a descrita na figura \ref{fig:assinatura}. Neste caso estou o info@ccc-centro.pt é uma \textit{mailing list} legítima pela qual eu iria receber o ataque.

Assim, este trabalho evidencia que pessoas das área de cibersegurança permanecem vulneráveis a ataques de \textit{spear phishing} altamente personalizados, especialmente quando exploram relações profissionais e contextos organizacionais familiares. A análise demonstra que a replicação de comunicações legítimas, aliada à criação de urgência temporal, constitui um vetor de ataque eficaz mesmo contra utilizadores tecnicamente preparados. A consciencialização contínua e a validação crítica de todas as comunicações são, portanto, essenciais para mitigar esta ameaça persistente.

\begin{thebibliography}{9}
	
	\bibitem{daengsi2021}
	T.~Daengsi, P.~Wuttidittachotti, P.~Pornpongtechavanich, e N.~Utakrit,
	``A comparative study of cybersecurity awareness on phishing among employees from different departments in an organization,''
	in \textit{Proceedings of the 2nd International Conference on Smart Computing and Electronic Enterprise (ICSCEE 2021)},
	IEEE, 2021.
	
	\bibitem{meyers2018}
	J.~J. Meyers, D.~L. Hansen, J.~S. Giboney, e D.~C. Rowe,
	``Training future cybersecurity professionals in spear phishing using SiEVE,''
	in \textit{The 19th Annual Conference on Information Technology Education (SIGITE'18)},
	2018.
	
\end{thebibliography}

\end{document}