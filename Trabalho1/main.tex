\documentclass[a4paper, titlepage, oneside, 11pt]{article}
\usepackage[total={6in, 8in}, margin=2.5cm, headsep=2cm, top=3cm ]{geometry}
\usepackage[portuguese]{babel}
\usepackage{graphicx}
\usepackage{parskip}
\usepackage[acronym]{glossaries}
\graphicspath{{images/}}
\usepackage{indentfirst}
\usepackage{setspace}
\usepackage{xcolor}
\definecolor{LightGray}{gray}{0.9}
\usepackage{float}
\usepackage{fontspec}
\usepackage{fancyvrb}
\usepackage{fancyhdr}
\pagestyle{fancy}
\usepackage{fvextra}
\usepackage{hyperref}
\usepackage{minted}
\usepackage{caption}
\hypersetup{colorlinks,linkcolor={black},citecolor={blue},urlcolor={blue}}
\setstretch{1.5}

\fancyheadoffset{0pt}

\fancyhead[L]{Psicologia e Cibersegurança \\[1ex] Trabalho 1\\} 
\fancyhead[R]{\includegraphics[width=0.3\linewidth]{images/deti.png}\\} 
\fancyfoot[C]{\thepage}
\fancyfoot[L]{João Nuno da Silva Luís}
\fancyfoot[R]{18/12/2025}


\usepackage{afterpage}
\usepackage{minted}

\newcommand\blankpage{%
    \null
    \thispagestyle{empty}%
    \addtocounter{page}{-1}%
    \newpage}

\newacronym{uc}{UC}{Unidade Curricular}
\newacronym{nis2}{NIS2}{Network and Information Security Directive 2}
\newacronym{grc}{GRC}{Governance, Risk and Compliance}
\newacronym{smd}{SMD}{Selo de Maturidade Digital}
\newacronym{rgpd}{RGPD}{Regulamento Geral de Proteção de Dados}

\begin{document}

\pagenumbering{arabic}

\section*{\centering{\Large{Se você conhece o inimigo e a si mesmo, não precisa temer o resultado de cem batalhas}}}

Neste trabalho pretende-se refletir sobre uma área na qual pretendo vir a especializar-me, o que é que fiz ou planeio fazer para alcançar esse \textit{expertise} na matéria em questão, qual a metodologia que penso usar para lá chegar e provas/evidências que mostrem o progresso feito nesse tema.

A área em que me pretendo especializar é na \acrfull{nis2}, que surge como um dos pilares fundamentais da estratégia europeia de cibersegurança, impondo novos requisitos legais, técnicos e organizacionais às entidades públicas e privadas e que muito recentemente, no dia quatro de dezembro, foi transposta para a legislação nacional através do Decreto-Lei nº125/2025. Assim, através da especialização na NIS2, desenvolverei também muito conhecimento numa das grandes áreas de Ciberseguranças que é \acrfull{grc}, ao combinar conhecimento técnico, enquadramento legal e responsabilidade organizacional.

\subsection*{A Diretiva NIS2: breve contextualização histórica e evolução}

A Diretiva NIS2 resulta da revisão da primeira Diretiva NIS, adotada em 2016, que constituiu o primeiro esforço legislativo europeu para estabelecer um nível comum de segurança das redes e sistemas de informação. Contudo, a rápida evolução das ameaças, o aumento da digitalização e a experiência adquirida com incidentes de grande impacto demonstraram que o regime anterior era insuficiente, tanto em termos de âmbito como de exigência.

Como resposta, a União Europeia atualizou a Diretiva para NIS2, com o objetivo de reforçar a resiliência cibernética, harmonizar requisitos entre Estados-Membros e assegurar uma maior responsabilização das organizações e das suas estruturas de gestão.

\clearpage

\subsection*{O que é a NIS2 e o que exige}

A NIS2 estabelece um quadro legal abrangente para a cibersegurança, com foco em três grandes pilares:

\begin{itemize}
	\item \textbf{Gestão de Risco:}
		As entidades abrangidas devem adotar medidas técnicas e organizacionais adequadas e proporcionais aos riscos identificados, incluindo políticas de segurança, controlo de acessos, gestão de vulnerabilidades, continuidade de negócio, backups, encriptação e formação dos colaboradores.
		Estas entidades abrangidas incluem setores como energia, transportes, saúde, água, serviços digitais mas também administração pública e unidades de investigação, afetando assim a Universidade de Aveiro.
	\item \textbf{Notificação de Incidentes:}
		A diretiva impõe obrigações rigorosas de notificação de incidentes significativos, introduzindo um modelo faseado de reporte que inclui um aviso inicial até 24h após a deteção do incidente, uma notificação mais detalhada até um máximo de 72h e depois um relatório final onde consta uma análise de impacto e lições aprendidas.
	\item \textbf{Governação e Responsabilidade:}
		Um dos aspetos mais relevantes da NIS2 é a responsabilização da gestão de topo. A administração das organizações passa a ter um papel ativo na supervisão da cibersegurança, podendo ser responsabilizada pessoalmente em caso de incumprimento. Há também um agravar das coimas, em caso de incumprimento.
\end{itemize}

\clearpage

Desta forma, ao conhecer mais a fundo a NIS2 e o que é que ela pede, desenvolvo conhecimento da área de \acrshort{grc} que se vai manter mesmo depois do prazo de implementação para o disposto no Decreto-Lei ter terminado, que são 2 anos.

De seguida, enuncio alguns tópicos que eu tenho aprendido e outros que tenho de conhecer mais a fundo para alcançar este objetivo:
 \begin{itemize}
 	\item Compliance com uma certificação, seja ela de âmbito mais nacional e direcionado para PMEs, como é o caso do \href{https://selosmaturidadedigital.incm.pt/}{Selo de Maturidade Digital} e a certificação de  serviços de cibersegurança lançado pelo CNCS \cite{cncs} ou de âmbito internacional, como é o caso da ISO27001 para a segurança da informação, a ISO20000 relativa a boas práticas nos serviços TI ou a ISO9001 para a qualidade dos processos
 	\item Aprender a interpretar e a tirar valor de \textit{frameworks} na área de Gestão de Risco de cibersegurança, como é o caso do CSF2.0 \cite{csf}, proposta pela NIST e que dá uma abordagem holística da aplicação da cibersegurança a toda a empresa
 	\item \acrfull{rgpd}, aprendido na Unidade Curricular de Direito e Organização da Segurança \cite{dos}, que faz parte do plano curricular do Mestrado em Cibersegurança
 	\item Cursos na \href{https://www.nau.edu.pt/pt/}{NAU}, como são os exemplos das figuras \ref{fig:CSIRT} e \ref{fig:GNS}
 \end{itemize}
 
Para além do acima mencionado, o conhecimento mais prático tem saído da aplicação na prática destes temas em conjunto com todo o tipo de entidades, sejam elas da administração pública, como escolas ou Câmaras Municipais ou PMEs dos vários setores de atividade, trabalho desenvolvido no Centro de Competências em Cibersegurança \cite{ccc} (website na figura \ref{fig:CCC}). 

\subsection*{Evidências}

\begin{figure}[H]
	\centering
	\includegraphics[width=0.9\linewidth]{images/NIS2}
	\caption{Certificado de participação na formação sobre NIS2 realizada pela empresa \textit{StrongStep}}
	\label{fig:NIS2}
\end{figure}

\begin{figure}[H]
	\centering
	\includegraphics[width=0.9\linewidth]{images/CSIRT}
	\caption{Certificado de formação sobre funcionamento de um CSIRT}
	\label{fig:CSIRT}
\end{figure}

\begin{figure}[H]
	\centering
	\includegraphics[width=0.9\linewidth]{images/CCC}
	\caption{Website do Centro de Competências em Cibersegurança da Região Centro (CCC-Centro)}
	\label{fig:CCC}
\end{figure}

\begin{figure}[H]
	\centering
	\includegraphics[width=0.9\linewidth]{images/GNS}
	\caption{Certificado de formação sobre a Segurança da Informação Classificada}
	\label{fig:GNS}
\end{figure}

\clearpage

\begin{thebibliography}{9}
	
	\bibitem{nis2}
	União Europeia,
	\textit{Diretiva (UE) 2022/2555 do Parlamento Europeu e do Conselho, de 14 de dezembro de 2022, relativa a medidas destinadas a garantir um elevado nível comum de cibersegurança na União (NIS2)},
	Jornal Oficial da União Europeia.
	Disponível em: \url{https://eur-lex.europa.eu/legal-content/PT/TXT/PDF/?uri=CELEX:32022L2555}
	
	\bibitem{dl1252025}
	Presidência do Conselho de Ministros,
	\textit{Decreto-Lei n.º 125/2025, de 4 de dezembro — Regime Jurídico da Cibersegurança},
	Diário da República, 1.ª série, n.º 234, 2025.
	Disponível em: \url{https://diariodarepublica.pt/dr/detalhe/decreto-lei/125-2025-962603401}
	
	\bibitem{csf}
	The NIST Cybersecurity Framework (CSF) 2.0.
	Disponível em: \url{https://nvlpubs.nist.gov/nistpubs/CSWP/NIST.CSWP.29.pdf}
	
	\bibitem{cncs}
	Certificação Serviços de Cibersegurança
	Disponível em: \url{https://www.cncs.gov.pt/pt/certificacao-servicos-de-ciberseguranca/}
	
	\bibitem{dos}
	Direito e Organização da Segurança
	Disponível em: \url{https://www.ua.pt/pt/uc/14826}
	
	\bibitem{ccc}
	Centro de Competências em Cibersegurança
	Disponível em: \url{https://www.ccc-centro.pt/}
	
\end{thebibliography}

\end{document}